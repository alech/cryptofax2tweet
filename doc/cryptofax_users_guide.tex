\documentclass[DIV14,11pt]{scrartcl}

\usepackage[T1]{fontenc}
\usepackage[scaled]{helvet}
\renewcommand*\familydefault{\sfdefault}

\usepackage{microtype}
\usepackage{xcolor}
\definecolor{darkblue}{rgb}{0 0 0.6}
\usepackage[colorlinks=true,urlcolor=darkblue]{hyperref}

\setlength{\parskip}{2ex plus 0.5ex minus 0.5ex}
\setlength{\parindent}{0mm}

\begin{document}


\title{CryptoFax2Tweet}
\author{Alexander Klink\\\texttt{\href{https://twitter.com/alech}{@alech}}\\\texttt{\href{mailto:cryptofax2tweet@alech.de}{cryptofax2tweet@alech.de}}}
\date{April 25th, 2011}

\maketitle

\thispagestyle{empty}
\section*{Using the CryptoFax2Tweet service}

The CryptoFax2Tweet service allows you to send a tweet via fax.
The tweet is encrypted using the RSA algorithm and encoded within a two-dimensional
	barcode.
It is intended for use when direct internet connections are unavailable to you,
	such as in the case of political turmoil or revolutions.
If you consider such a situation likely in the future, keeping the
\texttt{\href{http://private.chaos-darmstadt.de/~revolution/cryptofax.pdf}{cryptofax.pdf}} file at hand may prove helpful at a later time.
You are allowed and encouraged to pass on the file, too.

To use the service, open the file \texttt{cryptofax.pdf} in Adobe Reader
	(other PDF readers will most probably not work).
Reader will then ask you what you want to tweet.
Enter your tweet and send the resulting page with the barcode to the fax number
	displayed on the page.
The service on the other side will decode the barcode, decrypt the tweet and
send it to Twitter as the \texttt{\href{https://twitter.com/cryptofax2tweet}{@cryptofax2tweet}} user.

\section*{Security considerations}

While the tweet itself is encrypted on the page that you are sending out,
	someone monitoring both the phone line you are using and the
	\texttt{@cryptofax2tweet} Twitter user can gain information on what tweet
	you are sending by correlating the timing of the fax and the tweet.

If this is a matter of concern for you, make a call on either
	\emph{not using the service} or sending out the fax from a location as
	anonymous as possible (for example a call shop or similar).
If the page containing the barcode is intercepted before the fax is sent out,
	the information on the page is protected by the RSA encryption and can not
	be read by parties who do not have access to the RSA private key.

\section*{Further reading and Thanks}

% FIXME blog post
If you are interested in how this service works on a technical level, a blog
	post at FIXME provides more detailed information.

A thanks goes out to \href{https://as250.net}{AS250.net}, a non-profit organisation providing
	independent infrastructure for for Open Source and Community projects, which
	is making this service possible by running the fax to email gateway for it.
\end{document}
